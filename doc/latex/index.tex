\hypertarget{index_intro_sec}{}\section{Introduction}\label{index_intro_sec}
An extension for taskwarrior, that lets the user define a schedule for when certain contexts should be active.\hypertarget{index_intro_sub}{}\subsection{Ways to use this tool}\label{index_intro_sub}

\begin{DoxyItemize}
\item Define a schedule and let the program run in the background without interaction
\item Let the program notify you when certain actions are pending
\item Cancel active tasks automatically on a switch, in order to prevent tasks from getting excess length
\item Delay an imminent switch or cancel from happening with the delay command
\end{DoxyItemize}\hypertarget{index_install_sec}{}\section{How to install the tool}\label{index_install_sec}
platform\+: linux required libs\+: fcntl.\+h, getopt.\+h

make after successful build sudo make install\hypertarget{index_first_setup}{}\subsection{Initial setup}\label{index_first_setup}

\begin{DoxyItemize}
\item type csw, the program will create a folder csw and a file config within $\sim$/.task/
\item Enter the config file in $\sim$/.task/csw/config and setup your schedule Example\+:
\begin{DoxyItemize}
\item 5am to 8am wake up and study =$>$
\item Zone=wake-\/and-\/study;Start=05\+:00;End=08\+:00;context=study
\item 8\+:30am to 4pm work
\item Zone=Work;Start=08\+:30;End=16\+:00;context=work
\end{DoxyItemize}
\item make sure that the values entered at \textquotesingle{}context=\textquotesingle{} are equal to the assigned values at taskwarrior to see which contexts you have defined you can either
\begin{DoxyItemize}
\item look into $\sim$/.taskrc at the bottom
\item type task \+\_\+context into your terminal
\end{DoxyItemize}
\item If you want to exclude your schedule from certain weekdays enter the following line into the config\+:
\begin{DoxyItemize}
\item Exclude=permanent(sa,su) (Exclude every saturday and sunday) or
\item Exclude=temporary(2020-\/12-\/22\#2020-\/12-\/28) (Exclude from the 22nd december until the 28th december of 2020) or
\item Exclude=temporary(2020-\/08-\/12,2020-\/08-\/15) (Exclude the 12th and 15th august of 2020)
\end{DoxyItemize}
\end{DoxyItemize}\hypertarget{index_cronjob}{}\subsection{Cronjob}\label{index_cronjob}

\begin{DoxyItemize}
\item initially the program will create a cronjob with a 1 minute interval, if you desire a different interval\+:
\begin{DoxyItemize}
\item start the program with the -\/i \{M\+IN\} option
\item Example\+: \char`\"{}csw -\/i 3\char`\"{} install a 3 minute interval
\end{DoxyItemize}
\end{DoxyItemize}\hypertarget{index_cancel-notify}{}\subsection{Task cancel and notification}\label{index_cancel-notify}

\begin{DoxyItemize}
\item both options can be toggled with the -\/c \{1 = on$\vert$0 = off\} (cancel) and -\/n \{1 = on$\vert$0 = off\} option
\end{DoxyItemize}\hypertarget{index_delay}{}\subsection{Delay a switch and/or cancel from happening for a time-\/span}\label{index_delay}

\begin{DoxyItemize}
\item with the -\/d \{time-\/span\} option you can define a time a span until which no switch/cancel will occur
\begin{DoxyItemize}
\item Example time span for 12h\+: -\/d 720 -\/d 720min -\/d 12h -\/d 0.\+5d -\/d 720m -\/d 12hour
\item When a new delay is created upon an existing old one the new one is added to the old =$>$ 20min remaining on the old one , create new for 30min =$>$ 50min delay 
\end{DoxyItemize}
\end{DoxyItemize}